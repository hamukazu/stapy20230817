\documentclass[unicode,lualatex]{beamer}
\usepackage{luatexja}
\renewcommand{\kanjifamilydefault}{\gtdefault}
\usetheme[numbering=fraction]{metropolis}
\usepackage{hyperref}
\usepackage{xcolor}
\hypersetup{colorlinks,linkcolor=,urlcolor=blue}
\usepackage{geometry}
\usepackage{graphicx}
\title{{\tt math}と{\tt NumPy}の話}
\subtitle{みんなのPython勉強会資料}
\date{}
\author{加藤公一}
\begin{document}
\begin{frame}
 \titlepage
\end{frame}
\begin{frame}[fragile]{自己紹介}
\end{frame}
\begin{frame}[fragile]{今日の話}
  PythonのライブラリmathとNumPyについて
  \begin{itemize}
  \item mathってなに?NumPyってなに?
  \item mathとNumPyで何ができる?
  \item NumPyでできてmathでできないこと
  \item mathでできてNumPyでできないこと
  \item mathとNumPyの最近の進化
  \end{itemize}
\end{frame}
\begin{frame}[fragile]{mathとは?}
  オフィシャルサイトの説明:
  \begin{quote}
    This module provides access to the mathematical functions defined by the C standard.
  \end{quote}

  つまり、C言語の標準ライブラリmathで定義されている関数へのアクセスを提供する。

  {\tt math}はPythonの\emph{標準ライブラリ}(つまりPythonをインストールするとついてくる)
\end{frame}
\begin{frame}[fragile]{C言語のmathライブラリについて}
  C言語のmathライブラリの関数一覧はWikipedia参照:\newline
  {\tiny \url{https://en.wikipedia.org/wiki/C\_mathematical\_functions}}

$\cos(\pi)$をCとPythonで計算してみる。
\begin{minipage}[t]{0.45 \textwidth}
\fontsize{6pt}{6pt}\selectfont    
\noindent  
C言語
\begin{verbatim}
#include <math.h>
#include <stdio.h>

int main()
{
  printf("%f\n", cos(M_PI));
}
\end{verbatim}
\end{minipage}
\begin{minipage}[t]{0.45 \textwidth}
\fontsize{6pt}{6pt}\selectfont    
\noindent  
Python
\begin{verbatim}
import math

print(math.cos(math.pi))
\end{verbatim}
\end{minipage}
\end{frame}
\begin{frame}[fragile]{C言語のmathとPythonのmathの差分}
\noindent
PythonにあってCにないものの例
\begin{itemize}
\item \verb|factorial|: 階乗($n!$)を計算
\item \verb|comb|: ${}_nC_k$を計算
\item \verb|perm|: ${}_nP_k$を計算
\item \verb|nlp|:(あとで説明します)
\end{itemize}
CにあってPythonにないものの例
\begin{itemize}
\item \verb|div|: 整数の割り算で商とあまりを同時に計算する(Pythonは\verb|divmod|が同じ機能)
\end{itemize}
以上を見ると、PythonのmathはCのmathの機能をできるだけカバーしようとしているが、独自で便利そうな関数も揃えている。
\end{frame}
\begin{frame}[fragile]{NumPyとは?}
  オフィシャルサイトのキャッチフレーズ:
  \begin{quote}
    The fundamental package for scientific computing with Python
  \end{quote}

  配列(array)とその操作に関する機能が特徴的(これは\verb|math|にはない機能)


  mathに含まれる関数はほぼNumPyにも含まれている。
  (例:sin, cos, exp, log, ...)
\end{frame}
\begin{frame}[fragile]{NumPyの配列の機能1:Vectorization}
配列に数値を作用させたときに自動的にベクトル化される  
\fontsize{10pt}{10pt}\selectfont    
\begin{verbatim}
>>> import numpy as np
>>> a = np.array([1,2,3])
>>> b = 5
>>> a + b
array([6, 7, 8])
\end{verbatim}
\end{frame}
\begin{frame}[fragile]{NumPyの配列の機能2:Broadcasting}
配列の演算が要素ごとの演算として解釈される
\fontsize{10pt}{10pt}\selectfont    
\begin{verbatim}
>>> import numpy as np
>>> a = np.array([1,2,3])
>>> b = np.array([4,5,6])
>>> a + b
array([5, 7, 9])
>>> a * b
array([ 4, 10, 18])
\end{verbatim}
\end{frame}
\begin{frame}[fragile]{NumPyの配列の機能3:Indexing}
  複数のインデクスを一度に指定して要素を取り出せる
\fontsize{10pt}{10pt}\selectfont    
\begin{verbatim}
>>> import numpy as np
>>> idx = np.array([1,3,5])
>>> a = np.array([10,20,30,40,50,60])
>>> a[idx]
array([20, 40, 60])
\end{verbatim}  
\end{frame}
\begin{frame}[fragile]{mathとNumPyの使い分け}
  例えば、expを計算するときに{\tt math.exp}と{\tt numpy.exp}のどちらを使うべきか。\vspace{1cm}


結論:どちらでもよいし、あまり悩む必要ない。

ただし、
  \begin{itemize}
  \item mathよりNumPyのほうが高機能なので、NumPyが当然動くことが期待されている環境では、何も考えずNumPyを使ったほうがよいかも(判断コスト、思考コストの低減)
  \item NumPyは外部モジュールであることに注意。依存ライブラリも含めてリリースする場合(例えば、Windowsのインストーラを作るとき、AWS Lambdaで使うとき)は、不必要なのにNumPyを含めるのは無駄にファイルが大きくなる。
  \end{itemize}
\end{frame}
\begin{frame}[fragile]{ここまでのまとめ}
  \begin{itemize}
  \item PythonのmathはC言語のmathをカバーすることを目標としている
  \item NumPyのほうがmathより高機能。特に配列機能が特徴的。
  \item mathはNumPyのほぼサブセットなので、NumPyが使えるような環境では、mathは使わなくてもほぼNumPyだけで完結できる。
  \end{itemize}
\end{frame}
\begin{frame}
  「mathはNumPyのほぼサブセット」

  
  「ほぼ」とは?
\end{frame}
\begin{frame}[fragile]{mathはほんとにnumpyのサブセット?}
mathにあってnumpyにない関数の一覧を取得してみる。
{\fontsize{6pt}{6pt}\selectfont    
\begin{verbatim}
>>> import mathp
>>> import numpy as np
>>> math_funcs = set([x for x in dir(math) if not x.startswith("_")])
>>> np_funcs = set([x for x in dir(np) if not x.startswith("_")])
>>> sorted(math_funcs - np_funcs)
['acos', 'acosh', 'asin', 'asinh', 'atan', 'atan2', 'atanh', 'comb', 
'dist', 'erf', 'erfc', 'factorial', 'fsum', 'gamma', 'isqrt', 'lgamma',
'perm', 'pow', 'tau', 'ulp']
\end{verbatim}
}
それなりにある!

以下(網羅的ではなく)中身を見ていきます。
\end{frame}
\begin{frame}[fragile]{逆三角関数と逆双曲線関数}
  $\sin$の逆関数(数学では$\sin^{-1}$と書いたり、$\mathrm{arcsin}$と書いたりする)は、mathでは\verb|asin|であり、numpyでは\verb|arcsin|である。

  
  $\sinh$の逆関数(数学では$\sinh^{-1}$と書いたり、$\mathrm{arcsinh}$と書いたりする)は、mathでは\verb|asinh|であり、numpyでは\verb|arcsinh|である。

  以上、機能としては同じだが関数名が異なっている。
\end{frame}
\begin{frame}[fragile]{mathにあり、numpyにはないが、scipyにはあるもの}
  \verb|math.factorial| → \verb|scipy.special.factorial| ($n!$)
  
  \verb|math.comb| → \verb|scipy.special.comb| (${}_n C_k$)
  
  \verb|math.perm| → \verb|scipy.special.perm| (${}_n P_k$)

  \verb|math.gamma| → \verb|scipy.special.gamma| (ガンマ関数$\Gamma(x)$)
  
  \verb|math.lgamma| → \verb|scipy.special.loggamma| ($\log \Gamma(x)$)
\end{frame}
\begin{frame}[fragile]{{\tt math.ulp}}
  Pythonのドキュメントによる説明:
  {\small
  \begin{quote}
    \noindent math.ulp(x)
    
    \hspace{3mm}Return the value of the least significant bit of the float x:
  \end{quote}
}

least significant bitはunit in the last place(「最小変化単位」?日本語での一般的な訳はわからず)とも呼ばれ、その値を返す。
\end{frame}
\begin{frame}[fragile]{Unit in the last placeってなに?}
Wikipediaの解説:
\begin{figure}[!ht]
\includegraphics[scale=.4]{img/wikipedia-ulp.png}
\end{figure}
{\tiny \url{https://en.wikipedia.org/wiki/Unit\_in\_the\_last\_place}}

???

言い換えると、与えられた数に足すとコンピュータ内部表現としての数が変化するような数で最小のもの。

実行例:
{\fontsize{6pt}{6pt}\selectfont    
\begin{verbatim}
>>> import math
>>> math.ulp(1)
2.220446049250313e-16
>>> math.ulp(100)
1.4210854715202004e-14
>>> math.ulp(1e30)
140737488355328.0
\end{verbatim}
}
\end{frame}
\begin{frame}[fragile]{もっと詳しく知るために:浮動小数点数の内部表現}
  十進法の世界では、アボガドロ数:$6.02\times 10^{23}$のように、科学技術で使う数値は
  $\square.\square\square\square\square \times 10^{\square\square\square}$のようにされることが多い。

  特に仮数部の一の位がゼロにならないように調整する(例えば$6.02\times 10^{23} =
  0.602 \times 10^{24}$)
  
  二進法でも同様に$\square.\square\square\square\square_{(2)} \times 2^{\square\square\square}$で表す。

  二進法で仮数部の一の位がゼロでないように調整すると、一の位は1しかありえない。

\end{frame}
\begin{frame}[fragile]
  IEEE754では倍精度浮動小数点の内部表現を、符号1ビット、指数部11ビット、仮数部52ビットと規定している。
\begin{figure}[!ht]
\includegraphics[scale=.1]{img/{IEEE_754_Double_Floating_Point_Format.svg}.png}
\end{figure}

{\tiny 参照: \url{https://en.wikipedia.org/wiki/Double-precision\_floating-point\_format}}

符号は0:プラス、1:マイナス

指数部を符号なし11ビットとみなした数を$e$とすると、$e-1023$が指数となる($-1023$以上$1024$以下を表現できる)

仮数は一の位が1として、残りの52ビットを仮数部で表現する。

\[ (-1)^\mathrm{sign} \times (1.b_{51} b_{50} \cdots b_{0})_{(2)} \times 2^{e-1023} \]
\end{frame}
\begin{frame}[fragile]
  例:アボガドロ数(の近似値)$6.02\times 10^{23}$はコンピュータ内部表現ではどうなっているか。
{\fontsize{6pt}{6pt}\selectfont
\begin{verbatim}
>>> import struct
>>> a = struct.pack(">d",6.02e23) 
>>> bit_expression = ("").join([("0"*7+bin(b)[2:])[-8:] for b in a])
>>> bit_expression
'0100010011011111110111101001111100010000101010001101001101100001'
>>> bit_expression[0], bit_expression[1:12], bit_expression[12:]
('0', '10001001101', '1111110111101001111100010000101010001101001101100001')
\end{verbatim}
}
指数部$11111101_{(2)}=1101_{(10)}$なので、指数は$1101-1023=78$となり、つまり、
\[
  6.02\times 10^{23} = 1.111111011110100\cdots_{(2)} \times 2^{78}
\]

ほんとかな?検算してみよう!
{\fontsize{6pt}{6pt}\selectfont    
\begin{verbatim}
>>> 0b11111110111101001111100010000101010001101001101100001/2**52 * 2**78
6.02e+23
\end{verbatim}
}
\end{frame}
\begin{frame}[fragile]{もう一度:Unit in the last placeってなに?}
  与えられた数の、符号を正にして、仮数部を$0.0\dots 001_{(2)}$(小数第52位が1でほかは0)だと思ったもの。

  
  例:
  $6.02\times 10^{23}$の内部表現は、
  {\tiny \[\mathtt{1.1111110111101001111100010000101010001101001101100001}_{(2)} \times 2^{1101-1023}\]}
  これのULPは
  {\tiny \[\mathtt{0.0000000000000000000000000000000000000000000000000001}_{(2)} \times 2^{1101-1023}\]}

  したがって、ULPより小さい数を足しても、コンピュータ内部では影響がない。
\end{frame}
\end{document}
