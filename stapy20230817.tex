\documentclass[unicode,lualatex]{beamer}
\usepackage{luatexja}
\renewcommand{\kanjifamilydefault}{\gtdefault}
\usetheme[numbering=fraction]{metropolis}
\usepackage{hyperref}
\usepackage{xcolor}
\hypersetup{colorlinks,linkcolor=,urlcolor=blue}
\usepackage{geometry}
\title{{\tt math}と{\tt NumPy}の話}
\subtitle{みんなのPython勉強会資料}
\date{}
\author{加藤公一}
\begin{document}
\begin{frame}
 \titlepage
\end{frame}
\begin{frame}[fragile]{自己紹介}
\end{frame}
\begin{frame}[fragile]{今日の話}
  Pythonのライブラリ{\tt math}と{\tt NumPy}について
  \begin{itemize}
  \item {\tt math}ってなに?{\tt NumPy}ってなに?
  \item {\tt math}と{\tt NumPy}で何ができる?
  \item {\tt NumPy}でできて{\tt math}でできないこと
  \item {\tt math}でできて{\tt NumPy}でできないこと
  \item {\tt math}と{\tt NumPy}の最近の進化
  \end{itemize}
\end{frame}
\begin{frame}[fragile]{{\tt math}とは?}
  オフィシャルサイトの説明:
  \begin{quote}
    This module provides access to the mathematical functions defined by the C standard.
  \end{quote}

  つまり、C言語の標準ライブラリmathで定義されている関数へのアクセスを提供する。

  {\tt math}はPythonの\emph{標準ライブラリ}(つまりPythonをインストールするとついてくる)
\end{frame}
\begin{frame}[fragile]{C言語のmathライブラリについて}
  C言語のmathライブラリの関数一覧はWikipedia参照:\newline
  {\tiny \url{https://en.wikipedia.org/wiki/C_mathematical_functions}}

$\cos(\pi)$をCとPythonで計算してみる。
\begin{minipage}[t]{0.45 \textwidth}
\fontsize{6pt}{6pt}\selectfont    
\noindent  
C言語
\begin{verbatim}
#include <math.h>
#include <stdio.h>

int main()
{
  printf("%f\n", cos(M_PI));
}
\end{verbatim}
\end{minipage}
\begin{minipage}[t]{0.45 \textwidth}
\fontsize{6pt}{6pt}\selectfont    
\noindent  
Python
\begin{verbatim}
import math

print(math.cos(math.pi))
\end{verbatim}
\end{minipage}
\end{frame}
\begin{frame}[fragile]{C言語の{\tt math}とPythonの{\tt math}の差分}
\noindent
PythonにあってCにないものの例
\begin{itemize}
\item \verb|factorial|: 階乗($n!$)を計算
\item \verb|comb|: ${}_nC_k$を計算
\item \verb|perm|: ${}_nP_k$を計算
\item \verb|nlp|:(あとで説明します)
\end{itemize}
CにあってPythonにないものの例
\begin{itemize}
\item \verb|div|: 整数の割り算で商とあまりを同時に計算する(Pythonは\verb|divmod|が同じ機能)
\end{itemize}
以上を見ると、Pythonの\verb|math|はCの\verb|math|の機能をできるだけカバーしようとしているが、独自で便利そうな関数も揃えている。
\end{frame}
\begin{frame}[fragile]{{\tt NumPy}とは?}
  オフィシャルサイトのキャッチフレーズ:
  \begin{quote}
    The fundamental package for scientific computing with Python
  \end{quote}

  配列(array)とその操作に関する機能が特徴的(これは\verb|math|にはない機能)
\end{frame}
\begin{frame}[fragile]{{\tt NumPy}の配列でなにができるか}
\end{frame}
\end{document}
