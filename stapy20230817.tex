\documentclass[unicode,lualatex]{beamer}
\usepackage{luatexja}
\renewcommand{\kanjifamilydefault}{\gtdefault}
\usetheme[numbering=fraction]{metropolis}
\usepackage{hyperref}
\usepackage{xcolor}
\hypersetup{colorlinks,linkcolor=,urlcolor=blue}
\usepackage{geometry}
\title{{\tt math}と{\tt NumPy}の話}
\subtitle{みんなのPython勉強会資料}
\date{}
\author{加藤公一}
\begin{document}
\begin{frame}
 \titlepage
\end{frame}
\begin{frame}[fragile]{自己紹介}
\end{frame}
\begin{frame}[fragile]{今日の話}
  PythonのライブラリmathとNumPyについて
  \begin{itemize}
  \item mathってなに?NumPyってなに?
  \item mathとNumPyで何ができる?
  \item NumPyでできてmathでできないこと
  \item mathでできてNumPyでできないこと
  \item mathとNumPyの最近の進化
  \end{itemize}
\end{frame}
\begin{frame}[fragile]{mathとは?}
  オフィシャルサイトの説明:
  \begin{quote}
    This module provides access to the mathematical functions defined by the C standard.
  \end{quote}

  つまり、C言語の標準ライブラリmathで定義されている関数へのアクセスを提供する。

  {\tt math}はPythonの\emph{標準ライブラリ}(つまりPythonをインストールするとついてくる)
\end{frame}
\begin{frame}[fragile]{C言語のmathライブラリについて}
  C言語のmathライブラリの関数一覧はWikipedia参照:\newline
  {\tiny \url{https://en.wikipedia.org/wiki/C_mathematical_functions}}

$\cos(\pi)$をCとPythonで計算してみる。
\begin{minipage}[t]{0.45 \textwidth}
\fontsize{6pt}{6pt}\selectfont    
\noindent  
C言語
\begin{verbatim}
#include <math.h>
#include <stdio.h>

int main()
{
  printf("%f\n", cos(M_PI));
}
\end{verbatim}
\end{minipage}
\begin{minipage}[t]{0.45 \textwidth}
\fontsize{6pt}{6pt}\selectfont    
\noindent  
Python
\begin{verbatim}
import math

print(math.cos(math.pi))
\end{verbatim}
\end{minipage}
\end{frame}
\begin{frame}[fragile]{C言語のmathとPythonのmathの差分}
\noindent
PythonにあってCにないものの例
\begin{itemize}
\item \verb|factorial|: 階乗($n!$)を計算
\item \verb|comb|: ${}_nC_k$を計算
\item \verb|perm|: ${}_nP_k$を計算
\item \verb|nlp|:(あとで説明します)
\end{itemize}
CにあってPythonにないものの例
\begin{itemize}
\item \verb|div|: 整数の割り算で商とあまりを同時に計算する(Pythonは\verb|divmod|が同じ機能)
\end{itemize}
以上を見ると、PythonのmathはCのmathの機能をできるだけカバーしようとしているが、独自で便利そうな関数も揃えている。
\end{frame}
\begin{frame}[fragile]{NumPyとは?}
  オフィシャルサイトのキャッチフレーズ:
  \begin{quote}
    The fundamental package for scientific computing with Python
  \end{quote}

  配列(array)とその操作に関する機能が特徴的(これは\verb|math|にはない機能)


  mathに含まれる関数はほぼNumPyにも含まれている。
  (例:sin, cos, exp, log, ...)
\end{frame}
\begin{frame}[fragile]{NumPyの配列の機能1:Vectorization}
配列に数値を作用させたときに自動的にベクトル化される  
\fontsize{10pt}{10pt}\selectfont    
\begin{verbatim}
>>> import numpy as np
>>> a = np.array([1,2,3])
>>> b = 5
>>> a + b
array([6, 7, 8])
\end{verbatim}
\end{frame}
\begin{frame}[fragile]{NumPyの配列の機能2:Broadcasting}
配列の演算が要素ごとの演算として解釈される
\fontsize{10pt}{10pt}\selectfont    
\begin{verbatim}
>>> import numpy as np
>>> a = np.array([1,2,3])
>>> b = np.array([4,5,6])
>>> a + b
array([5, 7, 9])
>>> a * b
array([ 4, 10, 18])
\end{verbatim}
\end{frame}
\begin{frame}[fragile]{NumPyの配列の機能3:Indexing}
  複数のインデクスを一度に指定して要素を取り出せる
\fontsize{10pt}{10pt}\selectfont    
\begin{verbatim}
>>> import numpy as np
>>> idx = np.array([1,3,5])
>>> a = np.array([10,20,30,40,50,60])
>>> a[idx]
array([20, 40, 60])
\end{verbatim}  
\end{frame}
\begin{frame}[fragile]{mathとNumPyの使い分け}
  例えば、expを計算するときに{\tt math.exp}と{\tt numpy.exp}のどちらを使うべきか。\vspace{1cm}


結論:どちらでもよいし、あまり悩む必要ない。

ただし、
  \begin{itemize}
  \item mathよりNumPyのほうが高機能なので、NumPyが当然動くことが期待されている環境では、何も考えずNumPyを使ったほうがよいかも(判断コスト、思考コストの低減)
  \item NumPyは外部モジュールであることに注意。依存ライブラリも含めてリリースする場合(例えば、Windowsのインストーラを作るとき、AWS Lambdaで使うとき)は、不必要なのにNumPyを含めるのは無駄にファイルが大きくなる。
  \end{itemize}
\end{frame}
\begin{frame}[fragile]{ここまでのまとめ}
  \begin{itemize}
  \item PythonのmathはC言語のmathをカバーすることを目標としている
  \item NumPyのほうがmathより高機能。特に配列機能が特徴的。
  \item mathはNumPyのほぼサブセットなので、NumPyが使えるような環境では、mathは使わなくてもほぼNumPyだけで完結できる。
  \end{itemize}
\end{frame}
\begin{frame}
  「mathはNumPyのほぼサブセット」

  
  「ほぼ」とは?
\end{frame}
\end{document}
